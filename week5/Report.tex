\documentclass[11pt]{article}

    \usepackage[breakable]{tcolorbox}
    \usepackage{parskip} % Stop auto-indenting (to mimic markdown behaviour)
    
    \usepackage{iftex}
    \ifPDFTeX
    	\usepackage[T1]{fontenc}
    	\usepackage{mathpazo}
    \else
    	\usepackage{fontspec}
    \fi

    \usepackage{graphicx} %package to manage images
    \graphicspath{ {./images/} }
    \usepackage{wrapfig}
    
    % Basic figure setup, for now with no caption control since it's done
    % automatically by Pandoc (which extracts ![](path) syntax from Markdown).
    \usepackage{graphicx}
    % Maintain compatibility with old templates. Remove in nbconvert 6.0
    \let\Oldincludegraphics\includegraphics
    % Ensure that by default, figures have no caption (until we provide a
    % proper Figure object with a Caption API and a way to capture that
    % in the conversion process - todo).
    \usepackage{caption}
    \DeclareCaptionFormat{nocaption}{}
    \captionsetup{format=nocaption,aboveskip=0pt,belowskip=0pt}

    \usepackage[Export]{adjustbox} % Used to constrain images to a maximum size
    \adjustboxset{max size={0.9\linewidth}{0.9\paperheight}}
    \usepackage{float}
    \floatplacement{figure}{H} % forces figures to be placed at the correct location
    \usepackage{xcolor} % Allow colors to be defined
    \usepackage{enumerate} % Needed for markdown enumerations to work
    \usepackage{geometry} % Used to adjust the document margins
    \usepackage{amsmath} % Equations
    \usepackage{amssymb} % Equations
    \usepackage{textcomp} % defines textquotesingle
    % Hack from http://tex.stackexchange.com/a/47451/13684:
    \AtBeginDocument{%
        \def\PYZsq{\textquotesingle}% Upright quotes in Pygmentized code
    }
    \usepackage{upquote} % Upright quotes for verbatim code
    \usepackage{eurosym} % defines \euro
    \usepackage[mathletters]{ucs} % Extended unicode (utf-8) support
    \usepackage{fancyvrb} % verbatim replacement that allows latex
    \usepackage{grffile} % extends the file name processing of package graphics 
                         % to support a larger range
    \makeatletter % fix for grffile with XeLaTeX
    \def\Gread@@xetex#1{%
      \IfFileExists{"\Gin@base".bb}%
      {\Gread@eps{\Gin@base.bb}}%
      {\Gread@@xetex@aux#1}%
    }
    \makeatother

    % The hyperref package gives us a pdf with properly built
    % internal navigation ('pdf bookmarks' for the table of contents,
    % internal cross-reference links, web links for URLs, etc.)
    \usepackage{hyperref}
    % The default LaTeX title has an obnoxious amount of whitespace. By default,
    % titling removes some of it. It also provides customization options.
    \usepackage{titling}
    \usepackage{longtable} % longtable support required by pandoc >1.10
    \usepackage{booktabs}  % table support for pandoc > 1.12.2
    \usepackage[inline]{enumitem} % IRkernel/repr support (it uses the enumerate* environment)
    \usepackage[normalem]{ulem} % ulem is needed to support strikethroughs (\sout)
                                % normalem makes italics be italics, not underlines
    \usepackage{mathrsfs}
    

    
    % Colors for the hyperref package
    \definecolor{urlcolor}{rgb}{0,.145,.698}
    \definecolor{linkcolor}{rgb}{.71,0.21,0.01}
    \definecolor{citecolor}{rgb}{.12,.54,.11}

    % ANSI colors
    \definecolor{ansi-black}{HTML}{3E424D}
    \definecolor{ansi-black-intense}{HTML}{282C36}
    \definecolor{ansi-red}{HTML}{E75C58}
    \definecolor{ansi-red-intense}{HTML}{B22B31}
    \definecolor{ansi-green}{HTML}{00A250}
    \definecolor{ansi-green-intense}{HTML}{007427}
    \definecolor{ansi-yellow}{HTML}{DDB62B}
    \definecolor{ansi-yellow-intense}{HTML}{B27D12}
    \definecolor{ansi-blue}{HTML}{208FFB}
    \definecolor{ansi-blue-intense}{HTML}{0065CA}
    \definecolor{ansi-magenta}{HTML}{D160C4}
    \definecolor{ansi-magenta-intense}{HTML}{A03196}
    \definecolor{ansi-cyan}{HTML}{60C6C8}
    \definecolor{ansi-cyan-intense}{HTML}{258F8F}
    \definecolor{ansi-white}{HTML}{C5C1B4}
    \definecolor{ansi-white-intense}{HTML}{A1A6B2}
    \definecolor{ansi-default-inverse-fg}{HTML}{FFFFFF}
    \definecolor{ansi-default-inverse-bg}{HTML}{000000}

    % commands and environments needed by pandoc snippets
    % extracted from the output of `pandoc -s`
    \providecommand{\tightlist}{%
      \setlength{\itemsep}{0pt}\setlength{\parskip}{0pt}}
    \DefineVerbatimEnvironment{Highlighting}{Verbatim}{commandchars=\\\{\}}
    % Add ',fontsize=\small' for more characters per line
    \newenvironment{Shaded}{}{}
    \newcommand{\KeywordTok}[1]{\textcolor[rgb]{0.00,0.44,0.13}{\textbf{{#1}}}}
    \newcommand{\DataTypeTok}[1]{\textcolor[rgb]{0.56,0.13,0.00}{{#1}}}
    \newcommand{\DecValTok}[1]{\textcolor[rgb]{0.25,0.63,0.44}{{#1}}}
    \newcommand{\BaseNTok}[1]{\textcolor[rgb]{0.25,0.63,0.44}{{#1}}}
    \newcommand{\FloatTok}[1]{\textcolor[rgb]{0.25,0.63,0.44}{{#1}}}
    \newcommand{\CharTok}[1]{\textcolor[rgb]{0.25,0.44,0.63}{{#1}}}
    \newcommand{\StringTok}[1]{\textcolor[rgb]{0.25,0.44,0.63}{{#1}}}
    \newcommand{\CommentTok}[1]{\textcolor[rgb]{0.38,0.63,0.69}{\textit{{#1}}}}
    \newcommand{\OtherTok}[1]{\textcolor[rgb]{0.00,0.44,0.13}{{#1}}}
    \newcommand{\AlertTok}[1]{\textcolor[rgb]{1.00,0.00,0.00}{\textbf{{#1}}}}
    \newcommand{\FunctionTok}[1]{\textcolor[rgb]{0.02,0.16,0.49}{{#1}}}
    \newcommand{\RegionMarkerTok}[1]{{#1}}
    \newcommand{\ErrorTok}[1]{\textcolor[rgb]{1.00,0.00,0.00}{\textbf{{#1}}}}
    \newcommand{\NormalTok}[1]{{#1}}
    
    % Additional commands for more recent versions of Pandoc
    \newcommand{\ConstantTok}[1]{\textcolor[rgb]{0.53,0.00,0.00}{{#1}}}
    \newcommand{\SpecialCharTok}[1]{\textcolor[rgb]{0.25,0.44,0.63}{{#1}}}
    \newcommand{\VerbatimStringTok}[1]{\textcolor[rgb]{0.25,0.44,0.63}{{#1}}}
    \newcommand{\SpecialStringTok}[1]{\textcolor[rgb]{0.73,0.40,0.53}{{#1}}}
    \newcommand{\ImportTok}[1]{{#1}}
    \newcommand{\DocumentationTok}[1]{\textcolor[rgb]{0.73,0.13,0.13}{\textit{{#1}}}}
    \newcommand{\AnnotationTok}[1]{\textcolor[rgb]{0.38,0.63,0.69}{\textbf{\textit{{#1}}}}}
    \newcommand{\CommentVarTok}[1]{\textcolor[rgb]{0.38,0.63,0.69}{\textbf{\textit{{#1}}}}}
    \newcommand{\VariableTok}[1]{\textcolor[rgb]{0.10,0.09,0.49}{{#1}}}
    \newcommand{\ControlFlowTok}[1]{\textcolor[rgb]{0.00,0.44,0.13}{\textbf{{#1}}}}
    \newcommand{\OperatorTok}[1]{\textcolor[rgb]{0.40,0.40,0.40}{{#1}}}
    \newcommand{\BuiltInTok}[1]{{#1}}
    \newcommand{\ExtensionTok}[1]{{#1}}
    \newcommand{\PreprocessorTok}[1]{\textcolor[rgb]{0.74,0.48,0.00}{{#1}}}
    \newcommand{\AttributeTok}[1]{\textcolor[rgb]{0.49,0.56,0.16}{{#1}}}
    \newcommand{\InformationTok}[1]{\textcolor[rgb]{0.38,0.63,0.69}{\textbf{\textit{{#1}}}}}
    \newcommand{\WarningTok}[1]{\textcolor[rgb]{0.38,0.63,0.69}{\textbf{\textit{{#1}}}}}
    
    
    % Define a nice break command that doesn't care if a line doesn't already
    % exist.
    \def\br{\hspace*{\fill} \\* }
    % Math Jax compatibility definitions
    \def\gt{>}
    \def\lt{<}
    \let\Oldtex\TeX
    \let\Oldlatex\LaTeX
    \renewcommand{\TeX}{\textrm{\Oldtex}}
    \renewcommand{\LaTeX}{\textrm{\Oldlatex}}
    % Document parameters
    % Document title
    \title{2-factor-Hull-White\_model\_calibration}
    
    
    
    
    
% Pygments definitions
\makeatletter
\def\PY@reset{\let\PY@it=\relax \let\PY@bf=\relax%
    \let\PY@ul=\relax \let\PY@tc=\relax%
    \let\PY@bc=\relax \let\PY@ff=\relax}
\def\PY@tok#1{\csname PY@tok@#1\endcsname}
\def\PY@toks#1+{\ifx\relax#1\empty\else%
    \PY@tok{#1}\expandafter\PY@toks\fi}
\def\PY@do#1{\PY@bc{\PY@tc{\PY@ul{%
    \PY@it{\PY@bf{\PY@ff{#1}}}}}}}
\def\PY#1#2{\PY@reset\PY@toks#1+\relax+\PY@do{#2}}

\expandafter\def\csname PY@tok@w\endcsname{\def\PY@tc##1{\textcolor[rgb]{0.73,0.73,0.73}{##1}}}
\expandafter\def\csname PY@tok@c\endcsname{\let\PY@it=\textit\def\PY@tc##1{\textcolor[rgb]{0.25,0.50,0.50}{##1}}}
\expandafter\def\csname PY@tok@cp\endcsname{\def\PY@tc##1{\textcolor[rgb]{0.74,0.48,0.00}{##1}}}
\expandafter\def\csname PY@tok@k\endcsname{\let\PY@bf=\textbf\def\PY@tc##1{\textcolor[rgb]{0.00,0.50,0.00}{##1}}}
\expandafter\def\csname PY@tok@kp\endcsname{\def\PY@tc##1{\textcolor[rgb]{0.00,0.50,0.00}{##1}}}
\expandafter\def\csname PY@tok@kt\endcsname{\def\PY@tc##1{\textcolor[rgb]{0.69,0.00,0.25}{##1}}}
\expandafter\def\csname PY@tok@o\endcsname{\def\PY@tc##1{\textcolor[rgb]{0.40,0.40,0.40}{##1}}}
\expandafter\def\csname PY@tok@ow\endcsname{\let\PY@bf=\textbf\def\PY@tc##1{\textcolor[rgb]{0.67,0.13,1.00}{##1}}}
\expandafter\def\csname PY@tok@nb\endcsname{\def\PY@tc##1{\textcolor[rgb]{0.00,0.50,0.00}{##1}}}
\expandafter\def\csname PY@tok@nf\endcsname{\def\PY@tc##1{\textcolor[rgb]{0.00,0.00,1.00}{##1}}}
\expandafter\def\csname PY@tok@nc\endcsname{\let\PY@bf=\textbf\def\PY@tc##1{\textcolor[rgb]{0.00,0.00,1.00}{##1}}}
\expandafter\def\csname PY@tok@nn\endcsname{\let\PY@bf=\textbf\def\PY@tc##1{\textcolor[rgb]{0.00,0.00,1.00}{##1}}}
\expandafter\def\csname PY@tok@ne\endcsname{\let\PY@bf=\textbf\def\PY@tc##1{\textcolor[rgb]{0.82,0.25,0.23}{##1}}}
\expandafter\def\csname PY@tok@nv\endcsname{\def\PY@tc##1{\textcolor[rgb]{0.10,0.09,0.49}{##1}}}
\expandafter\def\csname PY@tok@no\endcsname{\def\PY@tc##1{\textcolor[rgb]{0.53,0.00,0.00}{##1}}}
\expandafter\def\csname PY@tok@nl\endcsname{\def\PY@tc##1{\textcolor[rgb]{0.63,0.63,0.00}{##1}}}
\expandafter\def\csname PY@tok@ni\endcsname{\let\PY@bf=\textbf\def\PY@tc##1{\textcolor[rgb]{0.60,0.60,0.60}{##1}}}
\expandafter\def\csname PY@tok@na\endcsname{\def\PY@tc##1{\textcolor[rgb]{0.49,0.56,0.16}{##1}}}
\expandafter\def\csname PY@tok@nt\endcsname{\let\PY@bf=\textbf\def\PY@tc##1{\textcolor[rgb]{0.00,0.50,0.00}{##1}}}
\expandafter\def\csname PY@tok@nd\endcsname{\def\PY@tc##1{\textcolor[rgb]{0.67,0.13,1.00}{##1}}}
\expandafter\def\csname PY@tok@s\endcsname{\def\PY@tc##1{\textcolor[rgb]{0.73,0.13,0.13}{##1}}}
\expandafter\def\csname PY@tok@sd\endcsname{\let\PY@it=\textit\def\PY@tc##1{\textcolor[rgb]{0.73,0.13,0.13}{##1}}}
\expandafter\def\csname PY@tok@si\endcsname{\let\PY@bf=\textbf\def\PY@tc##1{\textcolor[rgb]{0.73,0.40,0.53}{##1}}}
\expandafter\def\csname PY@tok@se\endcsname{\let\PY@bf=\textbf\def\PY@tc##1{\textcolor[rgb]{0.73,0.40,0.13}{##1}}}
\expandafter\def\csname PY@tok@sr\endcsname{\def\PY@tc##1{\textcolor[rgb]{0.73,0.40,0.53}{##1}}}
\expandafter\def\csname PY@tok@ss\endcsname{\def\PY@tc##1{\textcolor[rgb]{0.10,0.09,0.49}{##1}}}
\expandafter\def\csname PY@tok@sx\endcsname{\def\PY@tc##1{\textcolor[rgb]{0.00,0.50,0.00}{##1}}}
\expandafter\def\csname PY@tok@m\endcsname{\def\PY@tc##1{\textcolor[rgb]{0.40,0.40,0.40}{##1}}}
\expandafter\def\csname PY@tok@gh\endcsname{\let\PY@bf=\textbf\def\PY@tc##1{\textcolor[rgb]{0.00,0.00,0.50}{##1}}}
\expandafter\def\csname PY@tok@gu\endcsname{\let\PY@bf=\textbf\def\PY@tc##1{\textcolor[rgb]{0.50,0.00,0.50}{##1}}}
\expandafter\def\csname PY@tok@gd\endcsname{\def\PY@tc##1{\textcolor[rgb]{0.63,0.00,0.00}{##1}}}
\expandafter\def\csname PY@tok@gi\endcsname{\def\PY@tc##1{\textcolor[rgb]{0.00,0.63,0.00}{##1}}}
\expandafter\def\csname PY@tok@gr\endcsname{\def\PY@tc##1{\textcolor[rgb]{1.00,0.00,0.00}{##1}}}
\expandafter\def\csname PY@tok@ge\endcsname{\let\PY@it=\textit}
\expandafter\def\csname PY@tok@gs\endcsname{\let\PY@bf=\textbf}
\expandafter\def\csname PY@tok@gp\endcsname{\let\PY@bf=\textbf\def\PY@tc##1{\textcolor[rgb]{0.00,0.00,0.50}{##1}}}
\expandafter\def\csname PY@tok@go\endcsname{\def\PY@tc##1{\textcolor[rgb]{0.53,0.53,0.53}{##1}}}
\expandafter\def\csname PY@tok@gt\endcsname{\def\PY@tc##1{\textcolor[rgb]{0.00,0.27,0.87}{##1}}}
\expandafter\def\csname PY@tok@err\endcsname{\def\PY@bc##1{\setlength{\fboxsep}{0pt}\fcolorbox[rgb]{1.00,0.00,0.00}{1,1,1}{\strut ##1}}}
\expandafter\def\csname PY@tok@kc\endcsname{\let\PY@bf=\textbf\def\PY@tc##1{\textcolor[rgb]{0.00,0.50,0.00}{##1}}}
\expandafter\def\csname PY@tok@kd\endcsname{\let\PY@bf=\textbf\def\PY@tc##1{\textcolor[rgb]{0.00,0.50,0.00}{##1}}}
\expandafter\def\csname PY@tok@kn\endcsname{\let\PY@bf=\textbf\def\PY@tc##1{\textcolor[rgb]{0.00,0.50,0.00}{##1}}}
\expandafter\def\csname PY@tok@kr\endcsname{\let\PY@bf=\textbf\def\PY@tc##1{\textcolor[rgb]{0.00,0.50,0.00}{##1}}}
\expandafter\def\csname PY@tok@bp\endcsname{\def\PY@tc##1{\textcolor[rgb]{0.00,0.50,0.00}{##1}}}
\expandafter\def\csname PY@tok@fm\endcsname{\def\PY@tc##1{\textcolor[rgb]{0.00,0.00,1.00}{##1}}}
\expandafter\def\csname PY@tok@vc\endcsname{\def\PY@tc##1{\textcolor[rgb]{0.10,0.09,0.49}{##1}}}
\expandafter\def\csname PY@tok@vg\endcsname{\def\PY@tc##1{\textcolor[rgb]{0.10,0.09,0.49}{##1}}}
\expandafter\def\csname PY@tok@vi\endcsname{\def\PY@tc##1{\textcolor[rgb]{0.10,0.09,0.49}{##1}}}
\expandafter\def\csname PY@tok@vm\endcsname{\def\PY@tc##1{\textcolor[rgb]{0.10,0.09,0.49}{##1}}}
\expandafter\def\csname PY@tok@sa\endcsname{\def\PY@tc##1{\textcolor[rgb]{0.73,0.13,0.13}{##1}}}
\expandafter\def\csname PY@tok@sb\endcsname{\def\PY@tc##1{\textcolor[rgb]{0.73,0.13,0.13}{##1}}}
\expandafter\def\csname PY@tok@sc\endcsname{\def\PY@tc##1{\textcolor[rgb]{0.73,0.13,0.13}{##1}}}
\expandafter\def\csname PY@tok@dl\endcsname{\def\PY@tc##1{\textcolor[rgb]{0.73,0.13,0.13}{##1}}}
\expandafter\def\csname PY@tok@s2\endcsname{\def\PY@tc##1{\textcolor[rgb]{0.73,0.13,0.13}{##1}}}
\expandafter\def\csname PY@tok@sh\endcsname{\def\PY@tc##1{\textcolor[rgb]{0.73,0.13,0.13}{##1}}}
\expandafter\def\csname PY@tok@s1\endcsname{\def\PY@tc##1{\textcolor[rgb]{0.73,0.13,0.13}{##1}}}
\expandafter\def\csname PY@tok@mb\endcsname{\def\PY@tc##1{\textcolor[rgb]{0.40,0.40,0.40}{##1}}}
\expandafter\def\csname PY@tok@mf\endcsname{\def\PY@tc##1{\textcolor[rgb]{0.40,0.40,0.40}{##1}}}
\expandafter\def\csname PY@tok@mh\endcsname{\def\PY@tc##1{\textcolor[rgb]{0.40,0.40,0.40}{##1}}}
\expandafter\def\csname PY@tok@mi\endcsname{\def\PY@tc##1{\textcolor[rgb]{0.40,0.40,0.40}{##1}}}
\expandafter\def\csname PY@tok@il\endcsname{\def\PY@tc##1{\textcolor[rgb]{0.40,0.40,0.40}{##1}}}
\expandafter\def\csname PY@tok@mo\endcsname{\def\PY@tc##1{\textcolor[rgb]{0.40,0.40,0.40}{##1}}}
\expandafter\def\csname PY@tok@ch\endcsname{\let\PY@it=\textit\def\PY@tc##1{\textcolor[rgb]{0.25,0.50,0.50}{##1}}}
\expandafter\def\csname PY@tok@cm\endcsname{\let\PY@it=\textit\def\PY@tc##1{\textcolor[rgb]{0.25,0.50,0.50}{##1}}}
\expandafter\def\csname PY@tok@cpf\endcsname{\let\PY@it=\textit\def\PY@tc##1{\textcolor[rgb]{0.25,0.50,0.50}{##1}}}
\expandafter\def\csname PY@tok@c1\endcsname{\let\PY@it=\textit\def\PY@tc##1{\textcolor[rgb]{0.25,0.50,0.50}{##1}}}
\expandafter\def\csname PY@tok@cs\endcsname{\let\PY@it=\textit\def\PY@tc##1{\textcolor[rgb]{0.25,0.50,0.50}{##1}}}

\def\PYZbs{\char`\\}
\def\PYZus{\char`\_}
\def\PYZob{\char`\{}
\def\PYZcb{\char`\}}
\def\PYZca{\char`\^}
\def\PYZam{\char`\&}
\def\PYZlt{\char`\<}
\def\PYZgt{\char`\>}
\def\PYZsh{\char`\#}
\def\PYZpc{\char`\%}
\def\PYZdl{\char`\$}
\def\PYZhy{\char`\-}
\def\PYZsq{\char`\'}
\def\PYZdq{\char`\"}
\def\PYZti{\char`\~}
% for compatibility with earlier versions
\def\PYZat{@}
\def\PYZlb{[}
\def\PYZrb{]}
\makeatother


    % For linebreaks inside Verbatim environment from package fancyvrb. 
    \makeatletter
        \newbox\Wrappedcontinuationbox 
        \newbox\Wrappedvisiblespacebox 
        \newcommand*\Wrappedvisiblespace {\textcolor{red}{\textvisiblespace}} 
        \newcommand*\Wrappedcontinuationsymbol {\textcolor{red}{\llap{\tiny$\m@th\hookrightarrow$}}} 
        \newcommand*\Wrappedcontinuationindent {3ex } 
        \newcommand*\Wrappedafterbreak {\kern\Wrappedcontinuationindent\copy\Wrappedcontinuationbox} 
        % Take advantage of the already applied Pygments mark-up to insert 
        % potential linebreaks for TeX processing. 
        %        {, <, #, %, $, ' and ": go to next line. 
        %        _, }, ^, &, >, - and ~: stay at end of broken line. 
        % Use of \textquotesingle for straight quote. 
        \newcommand*\Wrappedbreaksatspecials {% 
            \def\PYGZus{\discretionary{\char`\_}{\Wrappedafterbreak}{\char`\_}}% 
            \def\PYGZob{\discretionary{}{\Wrappedafterbreak\char`\{}{\char`\{}}% 
            \def\PYGZcb{\discretionary{\char`\}}{\Wrappedafterbreak}{\char`\}}}% 
            \def\PYGZca{\discretionary{\char`\^}{\Wrappedafterbreak}{\char`\^}}% 
            \def\PYGZam{\discretionary{\char`\&}{\Wrappedafterbreak}{\char`\&}}% 
            \def\PYGZlt{\discretionary{}{\Wrappedafterbreak\char`\<}{\char`\<}}% 
            \def\PYGZgt{\discretionary{\char`\>}{\Wrappedafterbreak}{\char`\>}}% 
            \def\PYGZsh{\discretionary{}{\Wrappedafterbreak\char`\#}{\char`\#}}% 
            \def\PYGZpc{\discretionary{}{\Wrappedafterbreak\char`\%}{\char`\%}}% 
            \def\PYGZdl{\discretionary{}{\Wrappedafterbreak\char`\$}{\char`\$}}% 
            \def\PYGZhy{\discretionary{\char`\-}{\Wrappedafterbreak}{\char`\-}}% 
            \def\PYGZsq{\discretionary{}{\Wrappedafterbreak\textquotesingle}{\textquotesingle}}% 
            \def\PYGZdq{\discretionary{}{\Wrappedafterbreak\char`\"}{\char`\"}}% 
            \def\PYGZti{\discretionary{\char`\~}{\Wrappedafterbreak}{\char`\~}}% 
        } 
        % Some characters . , ; ? ! / are not pygmentized. 
        % This macro makes them "active" and they will insert potential linebreaks 
        \newcommand*\Wrappedbreaksatpunct {% 
            \lccode`\~`\.\lowercase{\def~}{\discretionary{\hbox{\char`\.}}{\Wrappedafterbreak}{\hbox{\char`\.}}}% 
            \lccode`\~`\,\lowercase{\def~}{\discretionary{\hbox{\char`\,}}{\Wrappedafterbreak}{\hbox{\char`\,}}}% 
            \lccode`\~`\;\lowercase{\def~}{\discretionary{\hbox{\char`\;}}{\Wrappedafterbreak}{\hbox{\char`\;}}}% 
            \lccode`\~`\:\lowercase{\def~}{\discretionary{\hbox{\char`\:}}{\Wrappedafterbreak}{\hbox{\char`\:}}}% 
            \lccode`\~`\?\lowercase{\def~}{\discretionary{\hbox{\char`\?}}{\Wrappedafterbreak}{\hbox{\char`\?}}}% 
            \lccode`\~`\!\lowercase{\def~}{\discretionary{\hbox{\char`\!}}{\Wrappedafterbreak}{\hbox{\char`\!}}}% 
            \lccode`\~`\/\lowercase{\def~}{\discretionary{\hbox{\char`\/}}{\Wrappedafterbreak}{\hbox{\char`\/}}}% 
            \catcode`\.\active
            \catcode`\,\active 
            \catcode`\;\active
            \catcode`\:\active
            \catcode`\?\active
            \catcode`\!\active
            \catcode`\/\active 
            \lccode`\~`\~ 	
        }
    \makeatother

    \let\OriginalVerbatim=\Verbatim
    \makeatletter
    \renewcommand{\Verbatim}[1][1]{%
        %\parskip\z@skip
        \sbox\Wrappedcontinuationbox {\Wrappedcontinuationsymbol}%
        \sbox\Wrappedvisiblespacebox {\FV@SetupFont\Wrappedvisiblespace}%
        \def\FancyVerbFormatLine ##1{\hsize\linewidth
            \vtop{\raggedright\hyphenpenalty\z@\exhyphenpenalty\z@
                \doublehyphendemerits\z@\finalhyphendemerits\z@
                \strut ##1\strut}%
        }%
        % If the linebreak is at a space, the latter will be displayed as visible
        % space at end of first line, and a continuation symbol starts next line.
        % Stretch/shrink are however usually zero for typewriter font.
        \def\FV@Space {%
            \nobreak\hskip\z@ plus\fontdimen3\font minus\fontdimen4\font
            \discretionary{\copy\Wrappedvisiblespacebox}{\Wrappedafterbreak}
            {\kern\fontdimen2\font}%
        }%
        
        % Allow breaks at special characters using \PYG... macros.
        \Wrappedbreaksatspecials
        % Breaks at punctuation characters . , ; ? ! and / need catcode=\active 	
        \OriginalVerbatim[#1,codes*=\Wrappedbreaksatpunct]%
    }
    \makeatother

    % Exact colors from NB
    \definecolor{incolor}{HTML}{303F9F}
    \definecolor{outcolor}{HTML}{D84315}
    \definecolor{cellborder}{HTML}{CFCFCF}
    \definecolor{cellbackground}{HTML}{F7F7F7}
    
    % prompt
    \makeatletter
    \newcommand{\boxspacing}{\kern\kvtcb@left@rule\kern\kvtcb@boxsep}
    \makeatother
    \newcommand{\prompt}[4]{
        \ttfamily\llap{{\color{#2}[#3]:\hspace{3pt}#4}}\vspace{-\baselineskip}
    }
    

    
    % Prevent overflowing lines due to hard-to-break entities
    \sloppy 
    % Setup hyperref package
    \hypersetup{
      breaklinks=true,  % so long urls are correctly broken across lines
      colorlinks=true,
      urlcolor=urlcolor,
      linkcolor=linkcolor,
      citecolor=citecolor,
      }
    % Slightly bigger margins than the latex defaults
    
    \geometry{verbose,tmargin=1in,bmargin=1in,lmargin=1in,rmargin=1in}
\begin{document}

\nocite{*} % this command forces all references in template.bib to be printed in the bibliography

\title{Price a Vanilla European Call Option}

\author{
  Ng, Joe Hoong\\
  \texttt{ng\_joehoong@hotmail.com}
  \and
  Nguyen, Dang Duy Nghia \\
  \texttt{nghia002@e.ntu.edu.sg}
   \and
  Ansari, Zain Us Sami Ahmed \\
  \texttt{zainussami@gmail.com}  
   \and
   Thorne, Dylan \\
  \texttt{dylan.thorne@gmail.com}  
  }

\date{April. 26, 2020} % if this is omitted, the current date is used for the title page
\maketitle

\noindent
\textbf{Keywords:} European Options, Heston model,  Constant Elasticity of Variance (CEV) model and Stochastic Volatility.



% the following creates an abstract -- it can be omitted
% an example of an environment: these have the form \begin{name} ... \end{name}
\begin{abstract}
    In this paper, we price a vanilla European call option under the Heston model and then simulate the monthly share price over a year using the Constant Elasticity of Variance (CEV) model, with the assumption of constant volatility each month. Monte Carlo simulations with varying sample sizes are run and the results are plotted against the closed form value for comparison.
\end{abstract}

\section{Introduction
}

In this paper we go beyond the constant volatility assumption in the Black-Scholes model.  Most of code implemented in this submission is derived from Module 5 \cite{M5} of the course. 

We initialize most variables as given by the question. 

\begin{itemize}
    \item Option maturity is one year
    \item The option is struck at-the-money
    \item The current share price is \$100
    \item The risk-free continuously compounded interest rate is 8\%
    \item The volatility for the underlying share is 30\%
\end{itemize}

\section{Fourier pricing technique under Heston model dynamics
}

Black-Scholes option pricing models assumed volatility of the underlying assets to be constant or a predetermined function of time, we will now implement a model which features instantaneous variance of asset price using volatility that evolves stochastically in time. Although there are several models incorporating stochastic volatility including introduced by Hull and White \cite{HW1}, Stein and Stein \cite{SS1} and Heston \cite{Heston1} \cite{Heston2}, we will implement Heston's constant interest rate model.

With the assumption that the underlying share follows the Heston model dynamics, the additional parameters required are specified as follows:
\begin{itemize}
    \item $\nu_0 = 0.06$
    \item $\kappa = 9$
    \item $\theta = 0.06$
    \item $\rho = -0.4$
\end{itemize}

    The Characteristic function is implemented using a function presented by
Albrecher et al \cite{AMST1}. The function is written as:
\[\phi_{S_T} = exp(C(\tau;u)+D(\tau;u)v_t + iu \log(S_t) )\]

Where,

\[C(\tau;u) = ri\tau u + \theta \kappa [\tau x\_ - \frac{1}{a}\log(\frac{1-ge^{d\tau}}{1-g})],\]

\[D(\tau;u) = (\frac{1-e^{d\tau}}{1-ge^{d\tau}})x\_,\]

\[\tau = T -t ,\]

\[ g = \frac{x\_}{x_+},\]

\[ x_\pm = \frac{b\pm d}{2a},\]

\[ d = \sqrt{b^2 - 4ac},\]

\[ c = - \frac{u^2 + ui}{2},\]

\[ b = \kappa - \rho \sigma iu,\]

\[ a = \frac {\sigma^2}{2}\]

    \begin{tcolorbox}[breakable, size=fbox, boxrule=1pt, pad at break*=1mm,colback=cellbackground, colframe=cellborder]
\prompt{In}{incolor}{}{\boxspacing}
\begin{Verbatim}[commandchars=\\\{\}]
\PY{c+c1}{\PYZsh{}Characteristic function code}

\PY{n}{a} \PY{o}{=} \PY{n}{sigma}\PY{o}{*}\PY{o}{*}\PY{l+m+mi}{2}\PY{o}{/}\PY{l+m+mi}{2}

\PY{k}{def} \PY{n+nf}{b}\PY{p}{(}\PY{n}{u}\PY{p}{)}\PY{p}{:}
    \PY{k}{return} \PY{n}{kappa} \PY{o}{\PYZhy{}} \PY{n}{rho}\PY{o}{*}\PY{n}{sigma}\PY{o}{*}\PY{l+m+mi}{1}\PY{n}{j}\PY{o}{*}\PY{n}{u}

\PY{k}{def} \PY{n+nf}{c}\PY{p}{(}\PY{n}{u}\PY{p}{)}\PY{p}{:}
    \PY{k}{return} \PY{o}{\PYZhy{}}\PY{p}{(}\PY{n}{u}\PY{o}{*}\PY{o}{*}\PY{l+m+mi}{2}\PY{o}{+}\PY{l+m+mi}{1}\PY{n}{j}\PY{o}{*}\PY{n}{u}\PY{p}{)}\PY{o}{/}\PY{l+m+mi}{2}

\PY{k}{def} \PY{n+nf}{d}\PY{p}{(}\PY{n}{u}\PY{p}{)}\PY{p}{:}
    \PY{k}{return} \PY{n}{np}\PY{o}{.}\PY{n}{sqrt}\PY{p}{(}\PY{n}{b}\PY{p}{(}\PY{n}{u}\PY{p}{)}\PY{o}{*}\PY{o}{*}\PY{l+m+mi}{2}\PY{o}{\PYZhy{}}\PY{l+m+mi}{4}\PY{o}{*}\PY{n}{a}\PY{o}{*}\PY{n}{c}\PY{p}{(}\PY{n}{u}\PY{p}{)}\PY{p}{)}

\PY{k}{def} \PY{n+nf}{xminus}\PY{p}{(}\PY{n}{u}\PY{p}{)}\PY{p}{:}
    \PY{k}{return} \PY{p}{(}\PY{n}{b}\PY{p}{(}\PY{n}{u}\PY{p}{)}\PY{o}{\PYZhy{}}\PY{n}{d}\PY{p}{(}\PY{n}{u}\PY{p}{)}\PY{p}{)}\PY{o}{/}\PY{p}{(}\PY{l+m+mi}{2}\PY{o}{*}\PY{n}{a}\PY{p}{)}

\PY{k}{def} \PY{n+nf}{xplus}\PY{p}{(}\PY{n}{u}\PY{p}{)}\PY{p}{:}
    \PY{k}{return} \PY{p}{(}\PY{n}{b}\PY{p}{(}\PY{n}{u}\PY{p}{)}\PY{o}{+}\PY{n}{d}\PY{p}{(}\PY{n}{u}\PY{p}{)}\PY{p}{)}\PY{o}{/}\PY{p}{(}\PY{l+m+mi}{2}\PY{o}{*}\PY{n}{a}\PY{p}{)}

\PY{k}{def} \PY{n+nf}{g}\PY{p}{(}\PY{n}{u}\PY{p}{)}\PY{p}{:}
    \PY{k}{return} \PY{n}{xminus}\PY{p}{(}\PY{n}{u}\PY{p}{)}\PY{o}{/}\PY{n}{xplus}\PY{p}{(}\PY{n}{u}\PY{p}{)}

\PY{k}{def} \PY{n+nf}{C}\PY{p}{(}\PY{n}{u}\PY{p}{)}\PY{p}{:}
    \PY{n}{val1} \PY{o}{=} \PY{n}{T}\PY{o}{*}\PY{n}{xminus}\PY{p}{(}\PY{n}{u}\PY{p}{)}\PY{o}{\PYZhy{}}\PY{n}{np}\PY{o}{.}\PY{n}{log}\PY{p}{(}\PY{p}{(}\PY{l+m+mi}{1}\PY{o}{\PYZhy{}}\PY{n}{g}\PY{p}{(}\PY{n}{u}\PY{p}{)}\PY{o}{*}\PY{n}{np}\PY{o}{.}\PY{n}{exp}\PY{p}{(}\PY{o}{\PYZhy{}}\PY{n}{T}\PY{o}{*}\PY{n}{d}\PY{p}{(}\PY{n}{u}\PY{p}{)}\PY{p}{)}\PY{p}{)}\PY{o}{/}\PY{p}{(}\PY{l+m+mi}{1}\PY{o}{\PYZhy{}}\PY{n}{g}\PY{p}{(}\PY{n}{u}\PY{p}{)}\PY{p}{)}\PY{p}{)}\PY{o}{/}\PY{n}{a}
    \PY{k}{return} \PY{n}{r}\PY{o}{*}\PY{n}{T}\PY{o}{*}\PY{l+m+mi}{1}\PY{n}{j}\PY{o}{*}\PY{n}{u} \PY{o}{+} \PY{n}{theta}\PY{o}{*}\PY{n}{kappa}\PY{o}{*}\PY{n}{val1}

\PY{k}{def} \PY{n+nf}{D}\PY{p}{(}\PY{n}{u}\PY{p}{)}\PY{p}{:}
    \PY{n}{val1} \PY{o}{=} \PY{l+m+mi}{1}\PY{o}{\PYZhy{}}\PY{n}{np}\PY{o}{.}\PY{n}{exp}\PY{p}{(}\PY{o}{\PYZhy{}}\PY{n}{T}\PY{o}{*}\PY{n}{d}\PY{p}{(}\PY{n}{u}\PY{p}{)}\PY{p}{)}
    \PY{n}{val2} \PY{o}{=} \PY{l+m+mi}{1}\PY{o}{\PYZhy{}}\PY{n}{g}\PY{p}{(}\PY{n}{u}\PY{p}{)}\PY{o}{*}\PY{n}{np}\PY{o}{.}\PY{n}{exp}\PY{p}{(}\PY{o}{\PYZhy{}}\PY{n}{T}\PY{o}{*}\PY{n}{d}\PY{p}{(}\PY{n}{u}\PY{p}{)}\PY{p}{)}
    \PY{k}{return} \PY{p}{(}\PY{n}{val1}\PY{o}{/}\PY{n}{val2}\PY{p}{)}\PY{o}{*}\PY{n}{xminus}\PY{p}{(}\PY{n}{u}\PY{p}{)}

\PY{k}{def} \PY{n+nf}{log\PYZus{}char}\PY{p}{(}\PY{n}{u}\PY{p}{)}\PY{p}{:}
    \PY{k}{return} \PY{n}{np}\PY{o}{.}\PY{n}{exp}\PY{p}{(}\PY{n}{C}\PY{p}{(}\PY{n}{u}\PY{p}{)} \PY{o}{+} \PY{n}{D}\PY{p}{(}\PY{n}{u}\PY{p}{)}\PY{o}{*}\PY{n}{v0} \PY{o}{+} \PY{l+m+mi}{1}\PY{n}{j}\PY{o}{*}\PY{n}{u}\PY{o}{*}\PY{n}{np}\PY{o}{.}\PY{n}{log}\PY{p}{(}\PY{n}{S0}\PY{p}{)}\PY{p}{)}

\PY{k}{def} \PY{n+nf}{adj\PYZus{}char}\PY{p}{(}\PY{n}{u}\PY{p}{)}\PY{p}{:}
    \PY{k}{return} \PY{n}{log\PYZus{}char}\PY{p}{(}\PY{n}{u}\PY{o}{\PYZhy{}}\PY{l+m+mi}{1}\PY{n}{j}\PY{p}{)}\PY{o}{/}\PY{n}{log\PYZus{}char}\PY{p}{(}\PY{o}{\PYZhy{}}\PY{l+m+mi}{1}\PY{n}{j}\PY{p}{)}
\end{Verbatim}
\end{tcolorbox}

    Now we vectorize the code, calculate an estimate for integrals and
calculate the Fourier estimate of our call price.

    \begin{tcolorbox}[breakable, size=fbox, boxrule=1pt, pad at break*=1mm,colback=cellbackground, colframe=cellborder]
\prompt{In}{incolor}{}{\boxspacing}
\begin{Verbatim}[commandchars=\\\{\}]
\PY{n}{delta\PYZus{}t} \PY{o}{=} \PY{n}{t\PYZus{}max}\PY{o}{/}\PY{n}{N}
\PY{n}{from\PYZus{}1\PYZus{}to\PYZus{}N} \PY{o}{=} \PY{n}{np}\PY{o}{.}\PY{n}{linspace}\PY{p}{(}\PY{l+m+mi}{1}\PY{p}{,}\PY{n}{N}\PY{p}{,}\PY{n}{N}\PY{p}{)}
\PY{n}{t\PYZus{}n} \PY{o}{=} \PY{p}{(}\PY{n}{from\PYZus{}1\PYZus{}to\PYZus{}N}\PY{o}{\PYZhy{}}\PY{l+m+mi}{1}\PY{o}{/}\PY{l+m+mi}{2}\PY{p}{)}\PY{o}{*}\PY{n}{delta\PYZus{}t}

\PY{c+c1}{\PYZsh{}Integral calculations}
\PY{n}{first\PYZus{}integral} \PY{o}{=} \PY{n+nb}{sum}\PY{p}{(}\PY{p}{(}\PY{p}{(}\PY{p}{(}\PY{n}{np}\PY{o}{.}\PY{n}{exp}\PY{p}{(}\PY{o}{\PYZhy{}}\PY{l+m+mi}{1}\PY{n}{j}\PY{o}{*}\PY{n}{t\PYZus{}n}\PY{o}{*}\PY{n}{k\PYZus{}log}\PY{p}{)}\PY{o}{*}\PY{n}{adj\PYZus{}char}\PY{p}{(}\PY{n}{t\PYZus{}n}\PY{p}{)}\PY{p}{)}\PY{o}{.}\PY{n}{imag}\PY{p}{)}\PY{o}{/}\PY{n}{t\PYZus{}n}\PY{p}{)}\PY{o}{*}\PY{n}{delta\PYZus{}t}\PY{p}{)}
\PY{n}{second\PYZus{}integral} \PY{o}{=} \PY{n+nb}{sum}\PY{p}{(}\PY{p}{(}\PY{p}{(}\PY{p}{(}\PY{n}{np}\PY{o}{.}\PY{n}{exp}\PY{p}{(}\PY{o}{\PYZhy{}}\PY{l+m+mi}{1}\PY{n}{j}\PY{o}{*}\PY{n}{t\PYZus{}n}\PY{o}{*}\PY{n}{k\PYZus{}log}\PY{p}{)}\PY{o}{*}\PY{n}{log\PYZus{}char}\PY{p}{(}\PY{n}{t\PYZus{}n}\PY{p}{)}\PY{p}{)}\PY{o}{.}\PY{n}{imag}\PY{p}{)}\PY{o}{/}\PY{n}{t\PYZus{}n}\PY{p}{)}\PY{o}{*}\PY{n}{delta\PYZus{}t}\PY{p}{)}

\PY{c+c1}{\PYZsh{}Call value}
\PY{n}{fourier\PYZus{}call\PYZus{}val} \PY{o}{=} \PY{n}{S0}\PY{o}{*}\PY{p}{(}\PY{l+m+mi}{1}\PY{o}{/}\PY{l+m+mi}{2} \PY{o}{+} \PY{n}{first\PYZus{}integral}\PY{o}{/}\PY{n}{np}\PY{o}{.}\PY{n}{pi}\PY{p}{)}\PY{o}{\PYZhy{}}\PY{n}{np}\PY{o}{.}\PY{n}{exp}\PY{p}{(}\PY{o}{\PYZhy{}}\PY{n}{r}\PY{o}{*}\PY{n}{T}\PY{p}{)}\PY{o}{*}\PY{n}{K}\PY{o}{*}\PY{p}{(}\PY{l+m+mi}{1}\PY{o}{/}\PY{l+m+mi}{2} \PY{o}{+} \PY{n}{second\PYZus{}integral}\PY{o}{/}\PY{n}{np}\PY{o}{.}\PY{n}{pi}\PY{p}{)}
\PY{n}{fourier\PYZus{}call\PYZus{}val}
\end{Verbatim}
\end{tcolorbox}

   
            \begin{tcolorbox}[breakable, size=fbox, boxrule=.5pt, pad at break*=1mm, opacityfill=0]
\prompt{Out}{outcolor}{}{\boxspacing}
\begin{Verbatim}[commandchars=\\\{\}]
13.734895692109077
\end{Verbatim}
\end{tcolorbox}

    To see the effectiveness of the pricing option under Heston dynamics we
will also price the call option under Black-Scholes assumption.

    \begin{tcolorbox}[breakable, size=fbox, boxrule=1pt, pad at break*=1mm,colback=cellbackground, colframe=cellborder]
\prompt{In}{incolor}{}{\boxspacing}
\begin{Verbatim}[commandchars=\\\{\}]
\PY{c+c1}{\PYZsh{} Code for analytical solution for vanilla European Call option}
\PY{n}{d\PYZus{}1\PYZus{}stock} \PY{o}{=} \PY{p}{(}\PY{n}{np}\PY{o}{.}\PY{n}{log}\PY{p}{(}\PY{n}{S0}\PY{o}{/}\PY{n}{K}\PY{p}{)}\PY{o}{+}\PY{p}{(}\PY{n}{r} \PY{o}{+} \PY{n}{sigma}\PY{o}{*}\PY{o}{*}\PY{l+m+mi}{2}\PY{o}{/}\PY{l+m+mi}{2}\PY{p}{)}\PY{o}{*}\PY{p}{(}\PY{n}{T}\PY{p}{)}\PY{p}{)}\PY{o}{/}\PY{p}{(}\PY{n}{sigma}\PY{o}{*}\PY{n}{np}\PY{o}{.}\PY{n}{sqrt}\PY{p}{(}\PY{n}{T}\PY{p}{)}\PY{p}{)}
\PY{n}{d\PYZus{}2\PYZus{}stock} \PY{o}{=} \PY{n}{d\PYZus{}1\PYZus{}stock} \PY{o}{\PYZhy{}} \PY{n}{sigma}\PY{o}{*}\PY{n}{np}\PY{o}{.}\PY{n}{sqrt}\PY{p}{(}\PY{n}{T}\PY{p}{)}

\PY{n}{analytic\PYZus{}callprice} \PY{o}{=} \PY{n}{S0}\PY{o}{*}\PY{n}{norm}\PY{o}{.}\PY{n}{cdf}\PY{p}{(}\PY{n}{d\PYZus{}1\PYZus{}stock}\PY{p}{)}\PY{o}{\PYZhy{}}\PY{n}{K}\PY{o}{*}\PY{n}{np}\PY{o}{.}\PY{n}{exp}\PY{p}{(}\PY{o}{\PYZhy{}}\PY{n}{r}\PY{o}{*}\PY{p}{(}\PY{n}{T}\PY{p}{)}\PY{p}{)}\PY{o}{*}\PY{n}{norm}\PY{o}{.}\PY{n}{cdf}\PY{p}{(}\PY{n}{d\PYZus{}2\PYZus{}stock}\PY{p}{)}
\PY{n}{analytic\PYZus{}callprice}
\end{Verbatim}
\end{tcolorbox}

            \begin{tcolorbox}[breakable, size=fbox, boxrule=.5pt, pad at break*=1mm, opacityfill=0]
\prompt{Out}{outcolor}{}{\boxspacing}
\begin{Verbatim}[commandchars=\\\{\}]
15.711312547892973
\end{Verbatim}
\end{tcolorbox}

\section{Simulate a share price path using CEV Model
}
Cox \cite{Cox1} developed the constant elasticity of variance (CEV) option pricing model, it attempts to capture stochastic volatility and is given by:

\[dS_t =  \mu S_t dt + \sigma  S^\gamma _t dW_t\]

If $\gamma = 1$ this model return the same value as Black-Scholes model, however if the value of $\gamma < 1 $ we experience an effect called leverage effect where the volatility increases as the price decreases over subsequent time periods.

Based on the assumption that $\sigma(t_i,t_{i+1}) = \sigma(S_{ti})^{\gamma-1}$, where $\sigma = 0.3$ and $\gamma = 0.75$.  We can simulate the next step in a share price path using the following formula:
$$S_{t_{i+1}} = S_{t_i}e^{(r - \frac{\sigma^2 (t_i,t_{i+1})}{2})(t_{i+1}-t_i)+\sigma(t_i, t_{i+1})\sqrt{t_{i+1}-t_i}Z}$$
where $S_{ti}$ is the share price at time $t_i$, $\sigma(t_i, t_{i+1})$ is the volatility for the period $[t_i,t_{i+1}]$, r is the risk-free interest rate, and Z~N(0, 1)

First, we define our helper functions. The next\_share\_price function
is used to calculate the evolution of the share price at t+1, from the
share price at t. We generate the random variable Z from within this
function. The effective sigma is also written as a function of the share
price at t.

Just for exploration purposes, we also added a varying\_vol flag, to
allow us to switch between a constant volatility and varying volatility.
We use the initial stock price instead of the previous price when
assuming a contant volatility.

The other function is the generate\_share\_price\_path function. We
first create an empty numpy array of shape (sample\_size x timesteps+1).
Note the addition of one element to the timestep, as the first element
is equal to the initial stock price. We then iterate through each path,
and each timestep, applying the next\_share\_price function against the
previous share price. We then convert the result into a pandas DataFrame
and return the results.

    \begin{tcolorbox}[breakable, size=fbox, boxrule=1pt, pad at break*=1mm,colback=cellbackground, colframe=cellborder]
\prompt{In}{incolor}{7}{\boxspacing}
\begin{Verbatim}[commandchars=\\\{\}]
\PY{k}{def} \PY{n+nf}{next\PYZus{}share\PYZus{}price}\PY{p}{(}\PY{n}{prev\PYZus{}price}\PY{p}{,} \PY{n}{r}\PY{p}{,} \PY{n}{dT}\PY{p}{,} \PY{n}{sigma\PYZus{}const}\PY{p}{,} \PY{n}{gamma}\PY{p}{,} \PY{n}{sample\PYZus{}size}\PY{p}{,} \PY{n}{varying\PYZus{}vol} \PY{o}{=} \PY{k+kc}{True}\PY{p}{)}\PY{p}{:}
    \PY{n}{Z} \PY{o}{=} \PY{n}{stats}\PY{o}{.}\PY{n}{norm}\PY{o}{.}\PY{n}{rvs}\PY{p}{(}\PY{n}{size}\PY{o}{=}\PY{n}{sample\PYZus{}size}\PY{p}{)}
    \PY{k}{if} \PY{n}{varying\PYZus{}vol}\PY{p}{:}
        \PY{n}{sigma} \PY{o}{=} \PY{n}{sigma\PYZus{}const}\PY{o}{*}\PY{p}{(}\PY{n}{prev\PYZus{}price}\PY{p}{)}\PY{o}{*}\PY{o}{*}\PY{p}{(}\PY{n}{gamma}\PY{o}{\PYZhy{}}\PY{l+m+mi}{1}\PY{p}{)}
    \PY{k}{else}\PY{p}{:}
        \PY{n}{sigma} \PY{o}{=} \PY{n}{sigma\PYZus{}const}\PY{o}{*}\PY{p}{(}\PY{n}{S0}\PY{p}{)}\PY{o}{*}\PY{o}{*}\PY{p}{(}\PY{n}{gamma}\PY{o}{\PYZhy{}}\PY{l+m+mi}{1}\PY{p}{)}
        
    \PY{k}{return} \PY{n}{prev\PYZus{}price}\PY{o}{*}\PY{n}{np}\PY{o}{.}\PY{n}{exp}\PY{p}{(}\PY{p}{(}\PY{n}{r}\PY{o}{\PYZhy{}}\PY{p}{(}\PY{n}{sigma}\PY{o}{*}\PY{o}{*}\PY{l+m+mi}{2}\PY{p}{)}\PY{o}{/}\PY{l+m+mi}{2}\PY{p}{)}\PY{o}{*}\PY{p}{(}\PY{n}{dT}\PY{p}{)}\PY{o}{+}\PY{p}{(}\PY{n}{sigma}\PY{p}{)}\PY{o}{*}\PY{p}{(}\PY{n}{np}\PY{o}{.}\PY{n}{sqrt}\PY{p}{(}\PY{n}{dT}\PY{p}{)}\PY{p}{)}\PY{o}{*}\PY{n}{Z}\PY{p}{)}

\PY{k}{def} \PY{n+nf}{generate\PYZus{}share\PYZus{}price\PYZus{}path}\PY{p}{(}\PY{n}{S0}\PY{p}{,} \PY{n}{r}\PY{p}{,} \PY{n}{T}\PY{p}{,} \PY{n}{sigma\PYZus{}const}\PY{p}{,} \PY{n}{gamma}\PY{p}{,} \PY{n}{sample\PYZus{}size}\PY{p}{,} \PY{n}{timesteps}\PY{p}{,} \PY{n}{varying\PYZus{}vol} \PY{o}{=} \PY{k+kc}{True}\PY{p}{)}\PY{p}{:}
    \PY{n}{df} \PY{o}{=} \PY{n}{pd}\PY{o}{.}\PY{n}{DataFrame}\PY{p}{(}\PY{p}{[}\PY{n}{S0}\PY{p}{]}\PY{o}{*}\PY{n}{sample\PYZus{}size}\PY{p}{)}
    \PY{k}{for} \PY{n}{t} \PY{o+ow}{in} \PY{n+nb}{range}\PY{p}{(}\PY{l+m+mi}{1}\PY{p}{,} \PY{n}{timesteps}\PY{o}{+}\PY{l+m+mi}{1}\PY{p}{)}\PY{p}{:}
        \PY{n}{df}\PY{p}{[}\PY{n}{t}\PY{p}{]} \PY{o}{=} \PY{n}{next\PYZus{}share\PYZus{}price}\PY{p}{(}\PY{n}{df}\PY{p}{[}\PY{n}{t}\PY{o}{\PYZhy{}}\PY{l+m+mi}{1}\PY{p}{]}\PY{p}{,} \PY{n}{r}\PY{p}{,} \PY{l+m+mi}{1}\PY{o}{/}\PY{n}{timesteps}\PY{p}{,} \PY{n}{sigma\PYZus{}const}\PY{p}{,} \PY{n}{gamma}\PY{p}{,} \PY{n}{sample\PYZus{}size}\PY{p}{,} \PY{n}{varying\PYZus{}vol}\PY{p}{)}
    \PY{k}{return} \PY{n}{df}\PY{o}{.}\PY{n}{T}
\end{Verbatim}
\end{tcolorbox}

    \begin{tcolorbox}[breakable, size=fbox, boxrule=1pt, pad at break*=1mm,colback=cellbackground, colframe=cellborder]
\prompt{In}{incolor}{8}{\boxspacing}
\begin{Verbatim}[commandchars=\\\{\}]
\PY{k+kn}{import} \PY{n+nn}{matplotlib}\PY{n+nn}{.}\PY{n+nn}{patches} \PY{k}{as} \PY{n+nn}{mpatches}
\PY{n}{T} \PY{o}{=} \PY{l+m+mi}{10}
\PY{n}{sample\PYZus{}size} \PY{o}{=} \PY{l+m+mi}{20}

\PY{n}{share\PYZus{}price\PYZus{}path\PYZus{}cev} \PY{o}{=} \PY{n}{generate\PYZus{}share\PYZus{}price\PYZus{}path}\PY{p}{(}\PY{n}{S0}\PY{p}{,} \PY{n}{r}\PY{p}{,} \PY{n}{T}\PY{p}{,} \PY{n}{sigma\PYZus{}const}\PY{p}{,} \PY{n}{gamma}\PY{p}{,} \PY{n}{sample\PYZus{}size}\PY{p}{,} \PY{n}{timesteps}\PY{p}{)}
\PY{n}{share\PYZus{}price\PYZus{}path\PYZus{}black\PYZus{}scholes} \PY{o}{=} \PY{n}{generate\PYZus{}share\PYZus{}price\PYZus{}path}\PY{p}{(}\PY{n}{S0}\PY{p}{,} \PY{n}{r}\PY{p}{,} \PY{n}{T}\PY{p}{,} \PY{n}{sigma\PYZus{}const}\PY{p}{,} \PY{l+m+mf}{1.0}\PY{p}{,} \PY{n}{sample\PYZus{}size}\PY{p}{,} \PY{n}{timesteps}\PY{p}{,} \PY{n}{varying\PYZus{}vol}\PY{o}{=}\PY{k+kc}{False}\PY{p}{)}

\PY{n}{plt}\PY{o}{.}\PY{n}{plot}\PY{p}{(}\PY{n}{share\PYZus{}price\PYZus{}path\PYZus{}cev}\PY{p}{,} \PY{n}{color}\PY{o}{=}\PY{l+s+s1}{\PYZsq{}}\PY{l+s+s1}{red}\PY{l+s+s1}{\PYZsq{}}\PY{p}{)}
\PY{n}{plt}\PY{o}{.}\PY{n}{plot}\PY{p}{(}\PY{n}{share\PYZus{}price\PYZus{}path\PYZus{}black\PYZus{}scholes}\PY{p}{,} \PY{n}{color}\PY{o}{=}\PY{l+s+s1}{\PYZsq{}}\PY{l+s+s1}{blue}\PY{l+s+s1}{\PYZsq{}}\PY{p}{)}
\PY{n}{plt}\PY{o}{.}\PY{n}{xlabel}\PY{p}{(}\PY{l+s+s2}{\PYZdq{}}\PY{l+s+s2}{Timestep}\PY{l+s+s2}{\PYZdq{}}\PY{p}{)}
\PY{n}{plt}\PY{o}{.}\PY{n}{ylabel}\PY{p}{(}\PY{l+s+s2}{\PYZdq{}}\PY{l+s+s2}{Share price}\PY{l+s+s2}{\PYZdq{}}\PY{p}{)}
\PY{n}{red\PYZus{}patch} \PY{o}{=} \PY{n}{mpatches}\PY{o}{.}\PY{n}{Patch}\PY{p}{(}\PY{n}{color}\PY{o}{=}\PY{l+s+s1}{\PYZsq{}}\PY{l+s+s1}{red}\PY{l+s+s1}{\PYZsq{}}\PY{p}{,} \PY{n}{label}\PY{o}{=}\PY{l+s+s1}{\PYZsq{}}\PY{l+s+s1}{CEV}\PY{l+s+s1}{\PYZsq{}}\PY{p}{)}
\PY{n}{blue\PYZus{}patch} \PY{o}{=} \PY{n}{mpatches}\PY{o}{.}\PY{n}{Patch}\PY{p}{(}\PY{n}{color}\PY{o}{=}\PY{l+s+s1}{\PYZsq{}}\PY{l+s+s1}{blue}\PY{l+s+s1}{\PYZsq{}}\PY{p}{,} \PY{n}{label}\PY{o}{=}\PY{l+s+s1}{\PYZsq{}}\PY{l+s+s1}{BSM}\PY{l+s+s1}{\PYZsq{}}\PY{p}{)}
\PY{n}{plt}\PY{o}{.}\PY{n}{legend}\PY{p}{(}\PY{n}{handles}\PY{o}{=}\PY{p}{[}\PY{n}{red\PYZus{}patch}\PY{p}{,} \PY{n}{blue\PYZus{}patch}\PY{p}{]}\PY{p}{,} \PY{n}{loc}\PY{o}{=}\PY{l+s+s1}{\PYZsq{}}\PY{l+s+s1}{upper left}\PY{l+s+s1}{\PYZsq{}}\PY{p}{)}
\PY{n}{plt}\PY{o}{.}\PY{n}{title}\PY{p}{(}\PY{l+s+s2}{\PYZdq{}}\PY{l+s+s2}{CEV vs BSM Stock Price Paths}\PY{l+s+s2}{\PYZdq{}}\PY{p}{)}
\PY{n}{plt}\PY{o}{.}\PY{n}{show}\PY{p}{(}\PY{p}{)}
\end{Verbatim}
\end{tcolorbox}

    \begin{center}
    \adjustimage{max size={0.9\linewidth}{0.9\paperheight}}{output_17_0.png}
    \end{center}
    { \hspace*{\fill} \\}

    Next, we create a python dictionary called share\_price\_paths, to hold
our results for part 2. The key of this dictionary would be the number
of sample price paths, while the values would be the dataframes
containing the price paths. We also track the rough processing time
required at each step, by printing the time when each iteration
completes.

    \begin{tcolorbox}[breakable, size=fbox, boxrule=1pt, pad at break*=1mm,colback=cellbackground, colframe=cellborder]
\prompt{In}{incolor}{9}{\boxspacing}
\begin{Verbatim}[commandchars=\\\{\}]
\PY{k+kn}{import} \PY{n+nn}{time}

\PY{n}{T} \PY{o}{=} \PY{l+m+mi}{1}
\PY{n}{sample\PYZus{}sizes} \PY{o}{=} \PY{n+nb}{range}\PY{p}{(}\PY{l+m+mi}{1000}\PY{p}{,} \PY{l+m+mi}{50001}\PY{p}{,} \PY{l+m+mi}{1000}\PY{p}{)}

\PY{n}{share\PYZus{}price\PYZus{}paths} \PY{o}{=} \PY{p}{\PYZob{}}\PY{p}{\PYZcb{}}

\PY{n+nb}{print}\PY{p}{(}\PY{l+s+s2}{\PYZdq{}}\PY{l+s+s2}{Start generating share price path}\PY{l+s+s2}{\PYZdq{}}\PY{p}{)}
\PY{n}{start} \PY{o}{=} \PY{n}{time}\PY{o}{.}\PY{n}{time}\PY{p}{(}\PY{p}{)}
\PY{k}{for} \PY{n}{sample\PYZus{}size} \PY{o+ow}{in} \PY{n}{sample\PYZus{}sizes}\PY{p}{:}
    \PY{n}{share\PYZus{}val} \PY{o}{=} \PY{n}{generate\PYZus{}share\PYZus{}price\PYZus{}path}\PY{p}{(}\PY{n}{S0}\PY{p}{,} \PY{n}{r}\PY{p}{,} \PY{n}{T}\PY{p}{,} \PY{n}{sigma\PYZus{}const}\PY{p}{,} \PY{n}{gamma}\PY{p}{,} \PY{n}{sample\PYZus{}size}\PY{p}{,} \PY{n}{timesteps}\PY{p}{,} \PY{n}{varying\PYZus{}vol}\PY{o}{=}\PY{k+kc}{False}\PY{p}{)} 
     
    \PY{n}{share\PYZus{}price\PYZus{}paths}\PY{p}{[}\PY{n}{sample\PYZus{}size}\PY{p}{]} \PY{o}{=} \PY{n}{share\PYZus{}val}
    \PY{c+c1}{\PYZsh{}print(\PYZdq{}Updated for sample size \PYZob{}\PYZcb{} at \PYZob{}\PYZcb{}\PYZdq{}.format(sample\PYZus{}size, datetime.datetime.now().strftime(\PYZsq{}\PYZpc{}H:\PYZpc{}M\PYZsq{})))}
\PY{n}{end} \PY{o}{=} \PY{n}{time}\PY{o}{.}\PY{n}{time}\PY{p}{(}\PY{p}{)}
\PY{n+nb}{print}\PY{p}{(}\PY{l+s+sa}{f}\PY{l+s+s2}{\PYZdq{}}\PY{l+s+s2}{Generating all samples paths takes }\PY{l+s+s2}{\PYZob{}}\PY{l+s+s2}{(end \PYZhy{} start):.2f\PYZcb{}s}\PY{l+s+s2}{\PYZdq{}}\PY{p}{)}
\end{Verbatim}
\end{tcolorbox}

    \begin{Verbatim}[commandchars=\\\{\}]
Start generating share price path
Generating all samples paths takes 2.58s
    \end{Verbatim}

    To display our output, we show the first 10 price paths generated by our
iteration with 1000 samples:

    \begin{tcolorbox}[breakable, size=fbox, boxrule=1pt, pad at break*=1mm,colback=cellbackground, colframe=cellborder]
\prompt{In}{incolor}{10}{\boxspacing}
\begin{Verbatim}[commandchars=\\\{\}]
\PY{n}{share\PYZus{}price\PYZus{}paths}\PY{p}{[}\PY{l+m+mi}{1000}\PY{p}{]}\PY{o}{.}\PY{n}{iloc}\PY{p}{[}\PY{p}{:}\PY{p}{,} \PY{l+m+mi}{0}\PY{p}{:}\PY{l+m+mi}{10}\PY{p}{]}
\end{Verbatim}
\end{tcolorbox}

            \begin{tcolorbox}[breakable, size=fbox, boxrule=.5pt, pad at break*=1mm, opacityfill=0]
\prompt{Out}{outcolor}{10}{\boxspacing}
\begin{Verbatim}[commandchars=\\\{\}]
             0           1           2           3           4           5  \textbackslash{}
0   100.000000  100.000000  100.000000  100.000000  100.000000  100.000000
1    99.626771   99.374526  100.689569   97.065445  106.862386   99.918343
2   103.745565   96.967926  102.622183   96.451414  106.332275   96.640428
3   107.154631   96.585243  105.974778   93.207852  100.177288   99.129417
4   111.716818   95.838855  101.540594   94.146336   98.646601  102.592136
5   120.620494   93.477848  100.930914   94.502805  100.662067  103.175996
6   122.217586   87.674152  101.799340   91.076602  101.876289  102.336328
7   123.138256   87.108925  103.904210   94.061221  106.060742  103.426255
8   129.359695   87.871374  100.925245   92.948402  115.010833   98.206543
9   123.575048   87.571414  105.883449   95.799071  117.769685   97.640631
10  125.486476   88.836040  104.337774   96.829970  119.866089  100.623917
11  124.538067   92.114326  106.173600   98.493852  118.497793   99.093147
12  123.548415   92.619346  104.478810   98.081928  114.719969   94.669329

             6           7           8           9
0   100.000000  100.000000  100.000000  100.000000
1   106.483031  100.796117  100.030802  102.593647
2   105.512423  101.627748   98.448810  103.179244
3   107.821291   98.518154   97.598741  103.563761
4   108.394844   97.918995   99.489053  110.865271
5   108.425203   99.326116   97.269739  108.916593
6   107.844372  103.199284   98.432527  108.089622
7   108.269775  100.286497  101.084604  107.860547
8   108.268731   99.113198  101.552266  103.792957
9   103.400905   98.076554  100.633537  106.623775
10  107.548810   98.062335   97.583078  110.004438
11  104.394412   98.481303   94.325492  107.980799
12  103.306627   95.171645   94.012006  108.960908
\end{Verbatim}
\end{tcolorbox}
        
    Next, we plot the price paths for the iteration with 1000 samples,

    \begin{tcolorbox}[breakable, size=fbox, boxrule=1pt, pad at break*=1mm,colback=cellbackground, colframe=cellborder]
\prompt{In}{incolor}{11}{\boxspacing}
\begin{Verbatim}[commandchars=\\\{\}]
\PY{n}{plt}\PY{o}{.}\PY{n}{plot}\PY{p}{(}\PY{n}{share\PYZus{}price\PYZus{}paths}\PY{p}{[}\PY{l+m+mi}{1000}\PY{p}{]}\PY{p}{)}
\PY{n}{plt}\PY{o}{.}\PY{n}{xlabel}\PY{p}{(}\PY{l+s+s1}{\PYZsq{}}\PY{l+s+s1}{Timestep}\PY{l+s+s1}{\PYZsq{}}\PY{p}{)}
\PY{n}{plt}\PY{o}{.}\PY{n}{ylabel}\PY{p}{(}\PY{l+s+s1}{\PYZsq{}}\PY{l+s+s1}{Share price}\PY{l+s+s1}{\PYZsq{}}\PY{p}{)}
\PY{n}{plt}\PY{o}{.}\PY{n}{title}\PY{p}{(}\PY{l+s+s1}{\PYZsq{}}\PY{l+s+s1}{Sample of 1000 Share Price Paths}\PY{l+s+s1}{\PYZsq{}}\PY{p}{)}
\PY{n}{plt}\PY{o}{.}\PY{n}{show}\PY{p}{(}\PY{p}{)}
\end{Verbatim}
\end{tcolorbox}

    \begin{center}
    \adjustimage{max size={0.9\linewidth}{0.9\paperheight}}{output_23_0.png}
    \end{center}
    { \hspace*{\fill} \\}

\section{Monte Carlo estimates}

    Using Monte Carlo, we calculate the price of the vanilla call option as
follows:

    \begin{tcolorbox}[breakable, size=fbox, boxrule=1pt, pad at break*=1mm,colback=cellbackground, colframe=cellborder]
\prompt{In}{incolor}{12}{\boxspacing}
\begin{Verbatim}[commandchars=\\\{\}]
\PY{n}{price\PYZus{}estimate} \PY{o}{=} \PY{p}{[}\PY{p}{]}
\PY{n}{price\PYZus{}std} \PY{o}{=} \PY{p}{[}\PY{p}{]}


\PY{k}{for} \PY{n}{size} \PY{o+ow}{in} \PY{n}{sample\PYZus{}sizes}\PY{p}{:}
    \PY{n}{S\PYZus{}Ts} \PY{o}{=} \PY{n}{share\PYZus{}price\PYZus{}paths}\PY{p}{[}\PY{n}{size}\PY{p}{]}\PY{o}{.}\PY{n}{iloc}\PY{p}{[}\PY{l+m+mi}{12}\PY{p}{,} \PY{p}{:}\PY{p}{]}
    \PY{n}{payoff} \PY{o}{=} \PY{n}{np}\PY{o}{.}\PY{n}{maximum}\PY{p}{(}\PY{n}{S\PYZus{}Ts} \PY{o}{\PYZhy{}} \PY{n}{K}\PY{p}{,} \PY{l+m+mi}{0}\PY{p}{)}
    \PY{n}{discounted\PYZus{}price} \PY{o}{=} \PY{n}{np}\PY{o}{.}\PY{n}{exp}\PY{p}{(}\PY{o}{\PYZhy{}}\PY{n}{r}\PY{o}{*}\PY{n}{T}\PY{p}{)}\PY{o}{*}\PY{n}{payoff}
    \PY{n}{price\PYZus{}estimate}\PY{o}{.}\PY{n}{append}\PY{p}{(}\PY{n}{discounted\PYZus{}price}\PY{o}{.}\PY{n}{mean}\PY{p}{(}\PY{p}{)}\PY{p}{)}
    \PY{n}{price\PYZus{}std}\PY{o}{.}\PY{n}{append}\PY{p}{(}\PY{n}{discounted\PYZus{}price}\PY{o}{.}\PY{n}{std}\PY{p}{(}\PY{p}{)}\PY{o}{/}\PY{n}{np}\PY{o}{.}\PY{n}{sqrt}\PY{p}{(}\PY{n}{size}\PY{p}{)}\PY{p}{)}
\end{Verbatim}
\end{tcolorbox}

    \begin{tcolorbox}[breakable, size=fbox, boxrule=1pt, pad at break*=1mm,colback=cellbackground, colframe=cellborder]
\prompt{In}{incolor}{13}{\boxspacing}
\begin{Verbatim}[commandchars=\\\{\}]
\PY{n+nb}{print}\PY{p}{(}\PY{l+s+s2}{\PYZdq{}}\PY{l+s+s2}{The price estimated by Monte Carlo when using sample size of 50,000 is : }\PY{l+s+si}{\PYZob{}:.3f\PYZcb{}}\PY{l+s+s2}{\PYZdq{}}\PY{o}{.}\PY{n}{format}\PY{p}{(}\PY{n}{price\PYZus{}estimate}\PY{p}{[}\PY{o}{\PYZhy{}}\PY{l+m+mi}{1}\PY{p}{]}\PY{p}{)}\PY{p}{)} 
\end{Verbatim}
\end{tcolorbox}

    \begin{Verbatim}[commandchars=\\\{\}]
The price estimated by Monte Carlo when using sample size of 50,000 is : 8.694
    \end{Verbatim}

We then compare with calculation of CEV model using noncentralchi-squared

    \begin{tcolorbox}[breakable, size=fbox, boxrule=1pt, pad at break*=1mm,colback=cellbackground, colframe=cellborder]
\prompt{In}{incolor}{14}{\boxspacing}
\begin{Verbatim}[commandchars=\\\{\}]
\PY{k+kn}{import} \PY{n+nn}{numpy} \PY{k}{as} \PY{n+nn}{np}
\PY{k+kn}{import} \PY{n+nn}{matplotlib}\PY{n+nn}{.}\PY{n+nn}{pyplot} \PY{k}{as} \PY{n+nn}{plt}
\PY{k+kn}{from} \PY{n+nn}{scipy}\PY{n+nn}{.}\PY{n+nn}{stats} \PY{k+kn}{import} \PY{n}{ncx2}

\PY{n}{S0} \PY{o}{=} \PY{l+m+mi}{100}
\PY{n}{sigma} \PY{o}{=} \PY{l+m+mf}{0.3}
\PY{n}{gamma} \PY{o}{=} \PY{l+m+mf}{0.75}
\PY{n}{r} \PY{o}{=} \PY{l+m+mf}{0.08}
\PY{n}{T} \PY{o}{=} \PY{l+m+mi}{1}
\end{Verbatim}
\end{tcolorbox}

    \begin{tcolorbox}[breakable, size=fbox, boxrule=1pt, pad at break*=1mm,colback=cellbackground, colframe=cellborder]
\prompt{In}{incolor}{15}{\boxspacing}
\begin{Verbatim}[commandchars=\\\{\}]
\PY{n}{z} \PY{o}{=} \PY{l+m+mi}{2} \PY{o}{+} \PY{l+m+mi}{1}\PY{o}{/}\PY{p}{(}\PY{l+m+mi}{1}\PY{o}{\PYZhy{}}\PY{n}{gamma}\PY{p}{)}
\PY{k}{def} \PY{n+nf}{C}\PY{p}{(}\PY{n}{t}\PY{p}{,}\PY{n}{K}\PY{p}{)}\PY{p}{:}
    \PY{n}{kappa} \PY{o}{=} \PY{l+m+mi}{2}\PY{o}{*}\PY{n}{r}\PY{o}{/}\PY{p}{(}\PY{n}{sigma}\PY{o}{*}\PY{o}{*}\PY{l+m+mi}{2}\PY{o}{*}\PY{p}{(}\PY{l+m+mi}{1}\PY{o}{\PYZhy{}}\PY{n}{gamma}\PY{p}{)}\PY{o}{*}\PY{p}{(}\PY{n}{np}\PY{o}{.}\PY{n}{exp}\PY{p}{(}\PY{l+m+mi}{2}\PY{o}{*}\PY{n}{r}\PY{o}{*}\PY{p}{(}\PY{l+m+mi}{1}\PY{o}{\PYZhy{}}\PY{n}{gamma}\PY{p}{)}\PY{o}{*}\PY{n}{t}\PY{p}{)}\PY{o}{\PYZhy{}}\PY{l+m+mi}{1}\PY{p}{)}\PY{p}{)}
    \PY{n}{x} \PY{o}{=} \PY{n}{kappa}\PY{o}{*}\PY{n}{S0}\PY{o}{*}\PY{o}{*}\PY{p}{(}\PY{l+m+mi}{2}\PY{o}{*}\PY{p}{(}\PY{l+m+mi}{1}\PY{o}{\PYZhy{}}\PY{n}{gamma}\PY{p}{)}\PY{p}{)}\PY{o}{*}\PY{n}{np}\PY{o}{.}\PY{n}{exp}\PY{p}{(}\PY{l+m+mi}{2}\PY{o}{*}\PY{n}{r}\PY{o}{*}\PY{p}{(}\PY{l+m+mi}{1}\PY{o}{\PYZhy{}}\PY{n}{gamma}\PY{p}{)}\PY{o}{*}\PY{n}{t}\PY{p}{)}
    \PY{n}{y} \PY{o}{=} \PY{n}{kappa}\PY{o}{*}\PY{n}{K}\PY{o}{*}\PY{o}{*}\PY{p}{(}\PY{l+m+mi}{2}\PY{o}{*}\PY{p}{(}\PY{l+m+mi}{1}\PY{o}{\PYZhy{}}\PY{n}{gamma}\PY{p}{)}\PY{p}{)}
    \PY{k}{return} \PY{n}{S0}\PY{o}{*}\PY{p}{(}\PY{l+m+mi}{1}\PY{o}{\PYZhy{}}\PY{n}{ncx2}\PY{o}{.}\PY{n}{cdf}\PY{p}{(}\PY{n}{y}\PY{p}{,}\PY{n}{z}\PY{p}{,}\PY{n}{x}\PY{p}{)}\PY{p}{)}\PY{o}{\PYZhy{}}\PY{n}{K}\PY{o}{*}\PY{n}{np}\PY{o}{.}\PY{n}{exp}\PY{p}{(}\PY{o}{\PYZhy{}}\PY{n}{r}\PY{o}{*}\PY{n}{t}\PY{p}{)}\PY{o}{*}\PY{n}{ncx2}\PY{o}{.}\PY{n}{cdf}\PY{p}{(}\PY{n}{x}\PY{p}{,}\PY{n}{z}\PY{o}{\PYZhy{}}\PY{l+m+mi}{2}\PY{p}{,}\PY{n}{y}\PY{p}{)}
\end{Verbatim}
\end{tcolorbox}

    \begin{tcolorbox}[breakable, size=fbox, boxrule=1pt, pad at break*=1mm,colback=cellbackground, colframe=cellborder]
\prompt{In}{incolor}{16}{\boxspacing}
\begin{Verbatim}[commandchars=\\\{\}]
\PY{n}{cev\PYZus{}call\PYZus{}price} \PY{o}{=} \PY{n}{C}\PY{p}{(}\PY{n}{T}\PY{p}{,} \PY{l+m+mi}{100}\PY{p}{)}
\end{Verbatim}
\end{tcolorbox}

    \begin{tcolorbox}[breakable, size=fbox, boxrule=1pt, pad at break*=1mm,colback=cellbackground, colframe=cellborder]
\prompt{In}{incolor}{17}{\boxspacing}
\begin{Verbatim}[commandchars=\\\{\}]
\PY{n+nb}{print}\PY{p}{(}\PY{l+s+s2}{\PYZdq{}}\PY{l+s+s2}{The price calculated via CEV model using noncentral chi\PYZhy{}squared distribution is : }\PY{l+s+si}{\PYZob{}:.3f\PYZcb{}}\PY{l+s+s2}{\PYZdq{}}\PY{o}{.}\PY{n}{format}\PY{p}{(}\PY{n}{cev\PYZus{}call\PYZus{}price}\PY{p}{)}\PY{p}{)}
\end{Verbatim}
\end{tcolorbox}

    \begin{Verbatim}[commandchars=\\\{\}]
The price calculated via CEV model using noncentral chi-squared distribution is
: 8.702
    \end{Verbatim}
    
    We plot the prices of our Monte Carlo estimates against the CEV
noncentral chi-squared distribution prices calculated from Part 1 above:

    \begin{tcolorbox}[breakable, size=fbox, boxrule=1pt, pad at break*=1mm,colback=cellbackground, colframe=cellborder]
\prompt{In}{incolor}{18}{\boxspacing}
\begin{Verbatim}[commandchars=\\\{\}]
\PY{n}{plt}\PY{o}{.}\PY{n}{plot}\PY{p}{(}\PY{p}{[}\PY{n}{cev\PYZus{}call\PYZus{}price}\PY{p}{]}\PY{o}{*}\PY{l+m+mi}{50}\PY{p}{,} \PY{n}{label}\PY{o}{=}\PY{l+s+s1}{\PYZsq{}}\PY{l+s+s1}{CEV Call Price}\PY{l+s+s1}{\PYZsq{}}\PY{p}{)}
\PY{n}{plt}\PY{o}{.}\PY{n}{plot}\PY{p}{(}\PY{n}{price\PYZus{}estimate}\PY{p}{,} \PY{l+s+s1}{\PYZsq{}}\PY{l+s+s1}{.}\PY{l+s+s1}{\PYZsq{}}\PY{p}{,} \PY{n}{label}\PY{o}{=}\PY{l+s+s1}{\PYZsq{}}\PY{l+s+s1}{Monte Carlo Estimates}\PY{l+s+s1}{\PYZsq{}}\PY{p}{)}
\PY{n}{plt}\PY{o}{.}\PY{n}{plot}\PY{p}{(}\PY{n}{price\PYZus{}estimate} \PY{o}{+} \PY{l+m+mi}{3}\PY{o}{*}\PY{n}{np}\PY{o}{.}\PY{n}{array}\PY{p}{(}\PY{n}{price\PYZus{}std}\PY{p}{)}\PY{p}{,} \PY{l+s+s1}{\PYZsq{}}\PY{l+s+s1}{r}\PY{l+s+s1}{\PYZsq{}}\PY{p}{)}
\PY{n}{plt}\PY{o}{.}\PY{n}{plot}\PY{p}{(}\PY{n}{price\PYZus{}estimate} \PY{o}{\PYZhy{}} \PY{l+m+mi}{3}\PY{o}{*}\PY{n}{np}\PY{o}{.}\PY{n}{array}\PY{p}{(}\PY{n}{price\PYZus{}std}\PY{p}{)}\PY{p}{,} \PY{l+s+s1}{\PYZsq{}}\PY{l+s+s1}{r}\PY{l+s+s1}{\PYZsq{}}\PY{p}{)}
\PY{n}{plt}\PY{o}{.}\PY{n}{legend}\PY{p}{(}\PY{p}{)}
\PY{n}{plt}\PY{o}{.}\PY{n}{show}\PY{p}{(}\PY{p}{)}
\end{Verbatim}
\end{tcolorbox}

    \begin{center}
    \adjustimage{max size={0.9\linewidth}{0.9\paperheight}}{output_35_0.png}
    \end{center}
    { \hspace*{\fill} \\}
    
    We also plot the prices of our Monte Carlo estimates against the Fourier
Call prices calculated from Part 1 above:

    \begin{tcolorbox}[breakable, size=fbox, boxrule=1pt, pad at break*=1mm,colback=cellbackground, colframe=cellborder]
\prompt{In}{incolor}{19}{\boxspacing}
\begin{Verbatim}[commandchars=\\\{\}]
\PY{n}{plt}\PY{o}{.}\PY{n}{plot}\PY{p}{(}\PY{p}{[}\PY{n}{fourier\PYZus{}call\PYZus{}val}\PY{p}{]}\PY{o}{*}\PY{l+m+mi}{50}\PY{p}{,} \PY{n}{label}\PY{o}{=}\PY{l+s+s1}{\PYZsq{}}\PY{l+s+s1}{Fourier Call Price}\PY{l+s+s1}{\PYZsq{}}\PY{p}{)}
\PY{n}{plt}\PY{o}{.}\PY{n}{plot}\PY{p}{(}\PY{n}{price\PYZus{}estimate}\PY{p}{,} \PY{l+s+s1}{\PYZsq{}}\PY{l+s+s1}{.}\PY{l+s+s1}{\PYZsq{}}\PY{p}{,} \PY{n}{label}\PY{o}{=}\PY{l+s+s1}{\PYZsq{}}\PY{l+s+s1}{Monte Carlo Estimates}\PY{l+s+s1}{\PYZsq{}}\PY{p}{)}
\PY{n}{plt}\PY{o}{.}\PY{n}{plot}\PY{p}{(}\PY{n}{price\PYZus{}estimate} \PY{o}{+} \PY{l+m+mi}{3}\PY{o}{*}\PY{n}{np}\PY{o}{.}\PY{n}{array}\PY{p}{(}\PY{n}{price\PYZus{}std}\PY{p}{)}\PY{p}{,} \PY{l+s+s1}{\PYZsq{}}\PY{l+s+s1}{r}\PY{l+s+s1}{\PYZsq{}}\PY{p}{)}
\PY{n}{plt}\PY{o}{.}\PY{n}{plot}\PY{p}{(}\PY{n}{price\PYZus{}estimate} \PY{o}{\PYZhy{}} \PY{l+m+mi}{3}\PY{o}{*}\PY{n}{np}\PY{o}{.}\PY{n}{array}\PY{p}{(}\PY{n}{price\PYZus{}std}\PY{p}{)}\PY{p}{,} \PY{l+s+s1}{\PYZsq{}}\PY{l+s+s1}{r}\PY{l+s+s1}{\PYZsq{}}\PY{p}{)}
\PY{n}{plt}\PY{o}{.}\PY{n}{legend}\PY{p}{(}\PY{p}{)}
\PY{n}{plt}\PY{o}{.}\PY{n}{show}\PY{p}{(}\PY{p}{)}
\end{Verbatim}
\end{tcolorbox}

    \begin{center}
    \adjustimage{max size={0.9\linewidth}{0.9\paperheight}}{output_37_0.png}
    \end{center}
    { \hspace*{\fill} \\}
    
    We notice that there is a discrepency. Upon further investigation, for
the Heston model, v\_0 = 0.06 (v\_0 = stock volatility \^{} 0.5), so
stock volatility = .06**0.5= 0.2449. Note that sigma under Heston model
refers to volatility of stock volatility

Under our stock price Monte Carlo calculation, the default stock
volatility is 0.0948 (given by 0.3\emph{(100)}-.25 ). Thus its much less
than Heston. To have the same initial stock volatility, we find a new
value for \(\sigma\) by equating
\(\sigma(S_{ti})^{\gamma-1} = \sqrt{0.06}\), giving us
\(\sigma = 0.775\).

We then find that our newly calculated Monte Carlo calculated call
prices are aligned with the Fourier prices.

    \begin{tcolorbox}[breakable, size=fbox, boxrule=1pt, pad at break*=1mm,colback=cellbackground, colframe=cellborder]
\prompt{In}{incolor}{20}{\boxspacing}
\begin{Verbatim}[commandchars=\\\{\}]
\PY{n}{sigma\PYZus{}const} \PY{o}{=} \PY{l+m+mf}{0.775}

\PY{n}{T} \PY{o}{=} \PY{l+m+mi}{1}
\PY{n}{sample\PYZus{}sizes} \PY{o}{=} \PY{n+nb}{range}\PY{p}{(}\PY{l+m+mi}{1000}\PY{p}{,} \PY{l+m+mi}{50001}\PY{p}{,} \PY{l+m+mi}{1000}\PY{p}{)}

\PY{n}{share\PYZus{}price\PYZus{}paths} \PY{o}{=} \PY{p}{\PYZob{}}\PY{p}{\PYZcb{}}

\PY{n+nb}{print}\PY{p}{(}\PY{l+s+s2}{\PYZdq{}}\PY{l+s+s2}{Start generating share price path}\PY{l+s+s2}{\PYZdq{}}\PY{p}{)}
\PY{n}{start} \PY{o}{=} \PY{n}{time}\PY{o}{.}\PY{n}{time}\PY{p}{(}\PY{p}{)}
\PY{k}{for} \PY{n}{sample\PYZus{}size} \PY{o+ow}{in} \PY{n}{sample\PYZus{}sizes}\PY{p}{:}
    \PY{n}{share\PYZus{}val} \PY{o}{=} \PY{n}{generate\PYZus{}share\PYZus{}price\PYZus{}path}\PY{p}{(}\PY{n}{S0}\PY{p}{,} \PY{n}{r}\PY{p}{,} \PY{n}{T}\PY{p}{,} \PY{n}{sigma\PYZus{}const}\PY{p}{,} \PY{n}{gamma}\PY{p}{,} \PY{n}{sample\PYZus{}size}\PY{p}{,} \PY{n}{timesteps}\PY{p}{,} \PY{n}{varying\PYZus{}vol}\PY{o}{=}\PY{k+kc}{False}\PY{p}{)} 
     
    \PY{n}{share\PYZus{}price\PYZus{}paths}\PY{p}{[}\PY{n}{sample\PYZus{}size}\PY{p}{]} \PY{o}{=} \PY{n}{share\PYZus{}val}
    \PY{c+c1}{\PYZsh{}print(\PYZdq{}Updated for sample size \PYZob{}\PYZcb{} at \PYZob{}\PYZcb{}\PYZdq{}.format(sample\PYZus{}size, datetime.datetime.now().strftime(\PYZsq{}\PYZpc{}H:\PYZpc{}M\PYZsq{})))}
\PY{n}{end} \PY{o}{=} \PY{n}{time}\PY{o}{.}\PY{n}{time}\PY{p}{(}\PY{p}{)}
\PY{n+nb}{print}\PY{p}{(}\PY{l+s+sa}{f}\PY{l+s+s2}{\PYZdq{}}\PY{l+s+s2}{Generating all samples paths takes }\PY{l+s+s2}{\PYZob{}}\PY{l+s+s2}{(end \PYZhy{} start):.2f\PYZcb{}s}\PY{l+s+s2}{\PYZdq{}}\PY{p}{)}
\end{Verbatim}
\end{tcolorbox}

    \begin{Verbatim}[commandchars=\\\{\}]
Start generating share price path
Generating all samples paths takes 2.10s
    \end{Verbatim}

    \begin{tcolorbox}[breakable, size=fbox, boxrule=1pt, pad at break*=1mm,colback=cellbackground, colframe=cellborder]
\prompt{In}{incolor}{21}{\boxspacing}
\begin{Verbatim}[commandchars=\\\{\}]
\PY{n}{price\PYZus{}estimate} \PY{o}{=} \PY{p}{[}\PY{p}{]}
\PY{n}{price\PYZus{}std} \PY{o}{=} \PY{p}{[}\PY{p}{]}


\PY{k}{for} \PY{n}{size} \PY{o+ow}{in} \PY{n}{sample\PYZus{}sizes}\PY{p}{:}
    \PY{n}{S\PYZus{}Ts} \PY{o}{=} \PY{n}{share\PYZus{}price\PYZus{}paths}\PY{p}{[}\PY{n}{size}\PY{p}{]}\PY{o}{.}\PY{n}{iloc}\PY{p}{[}\PY{l+m+mi}{12}\PY{p}{,} \PY{p}{:}\PY{p}{]}
    \PY{n}{payoff} \PY{o}{=} \PY{n}{np}\PY{o}{.}\PY{n}{maximum}\PY{p}{(}\PY{n}{S\PYZus{}Ts} \PY{o}{\PYZhy{}} \PY{n}{K}\PY{p}{,} \PY{l+m+mi}{0}\PY{p}{)}
    \PY{n}{discounted\PYZus{}price} \PY{o}{=} \PY{n}{np}\PY{o}{.}\PY{n}{exp}\PY{p}{(}\PY{o}{\PYZhy{}}\PY{n}{r}\PY{o}{*}\PY{n}{T}\PY{p}{)}\PY{o}{*}\PY{n}{payoff}
    \PY{n}{price\PYZus{}estimate}\PY{o}{.}\PY{n}{append}\PY{p}{(}\PY{n}{discounted\PYZus{}price}\PY{o}{.}\PY{n}{mean}\PY{p}{(}\PY{p}{)}\PY{p}{)}
    \PY{n}{price\PYZus{}std}\PY{o}{.}\PY{n}{append}\PY{p}{(}\PY{n}{discounted\PYZus{}price}\PY{o}{.}\PY{n}{std}\PY{p}{(}\PY{p}{)}\PY{o}{/}\PY{n}{np}\PY{o}{.}\PY{n}{sqrt}\PY{p}{(}\PY{n}{size}\PY{p}{)}\PY{p}{)}
\end{Verbatim}
\end{tcolorbox}

    \begin{tcolorbox}[breakable, size=fbox, boxrule=1pt, pad at break*=1mm,colback=cellbackground, colframe=cellborder]
\prompt{In}{incolor}{22}{\boxspacing}
\begin{Verbatim}[commandchars=\\\{\}]
\PY{n}{plt}\PY{o}{.}\PY{n}{plot}\PY{p}{(}\PY{p}{[}\PY{n}{fourier\PYZus{}call\PYZus{}val}\PY{p}{]}\PY{o}{*}\PY{l+m+mi}{50}\PY{p}{,} \PY{n}{label}\PY{o}{=}\PY{l+s+s1}{\PYZsq{}}\PY{l+s+s1}{Fourier Call Price}\PY{l+s+s1}{\PYZsq{}}\PY{p}{)}
\PY{n}{plt}\PY{o}{.}\PY{n}{plot}\PY{p}{(}\PY{n}{price\PYZus{}estimate}\PY{p}{,} \PY{l+s+s1}{\PYZsq{}}\PY{l+s+s1}{.}\PY{l+s+s1}{\PYZsq{}}\PY{p}{,} \PY{n}{label}\PY{o}{=}\PY{l+s+s1}{\PYZsq{}}\PY{l+s+s1}{Monte Carlo Estimates}\PY{l+s+s1}{\PYZsq{}}\PY{p}{)}
\PY{n}{plt}\PY{o}{.}\PY{n}{plot}\PY{p}{(}\PY{n}{price\PYZus{}estimate} \PY{o}{+} \PY{l+m+mi}{3}\PY{o}{*}\PY{n}{np}\PY{o}{.}\PY{n}{array}\PY{p}{(}\PY{n}{price\PYZus{}std}\PY{p}{)}\PY{p}{,} \PY{l+s+s1}{\PYZsq{}}\PY{l+s+s1}{r}\PY{l+s+s1}{\PYZsq{}}\PY{p}{)}
\PY{n}{plt}\PY{o}{.}\PY{n}{plot}\PY{p}{(}\PY{n}{price\PYZus{}estimate} \PY{o}{\PYZhy{}} \PY{l+m+mi}{3}\PY{o}{*}\PY{n}{np}\PY{o}{.}\PY{n}{array}\PY{p}{(}\PY{n}{price\PYZus{}std}\PY{p}{)}\PY{p}{,} \PY{l+s+s1}{\PYZsq{}}\PY{l+s+s1}{r}\PY{l+s+s1}{\PYZsq{}}\PY{p}{)}
\PY{n}{plt}\PY{o}{.}\PY{n}{legend}\PY{p}{(}\PY{p}{)}
\PY{n}{plt}\PY{o}{.}\PY{n}{show}\PY{p}{(}\PY{p}{)}
\end{Verbatim}
\end{tcolorbox}

    \begin{center}
    \adjustimage{max size={0.9\linewidth}{0.9\paperheight}}{output_41_0.png}
    \end{center}
    { \hspace*{\fill} \\}    

\section{Conclusion}

A vanilla European call option was priced for a fluctuating volatility condition, using the Heston model. The call price was evaluated as \$13.73, which was verified to be similar to - but lower than - a constant volatility estimate using the Black-Scholes model at \$15.71.

Initially the values calculated for the underlying share price did not agree as expected, with the estimates well outside the expected error range. This was because the volatility term was not equivalent in the two different calculation methods. After resolving this, the CEV model calculated a price of \$8.70 for the underlying share, which agreed closely with the Monte Carlo estimate of \$8.69. This represents an absolute error of 1c and a relative discrepancy of approximately 0.1%.

The agreement between the results from various methods, although expected, provides confidence to choose the most suitable method for a situation with the knowledge that the results are accurate within a small margin of error.

\newpage
\section*{} \label{bibsection}


% the second parameter MMMMM should be as long as the longest label you use, in my case Smoller -- if you use % numbers only, use 99
% use \cite{refname} to refer to bibliography item \bibitem{refname} 
% LaTeX assigns a number, unless you use \bibitem[Name]{refname} -- in this case
% LaTeX prints Name when you use \cite{refname}
\begin{thebibliography}{MMMMM} 
\bibitem{HW1} Hull, J. and White, A. (1987). The pricing of options on assets
with stochastic volatilities. The journal of finance, 42(2):281--300.
\bibitem{SS1} Stein, E. M. and Stein, J. C. (1991). Stock price distributions
with stochastic volatility: an analytic approach. Review of financial
Studies, 4(4):727--752.
\bibitem{Heston1} Heston, S. L. (1993). A closed-form solution for options with
stochastic volatility with applications to bond and currency options.
Review of financial studies, 6(2):327--343.
\bibitem{Heston2} Heston, S. L. (1997). A simple new formula for options with
stochastic volatility.
\bibitem{AMST1} Albrecher, H., Mayer, P., Schoutens, W. and Tistaert, J. (2007).
``The Little Heston Trap'', Wilmott (1): 83--92.
\bibitem{Cox1}  Cox, John. "Notes on option pricing I: Constant elasticity of variance diffusions." Unpublished note, Stanford University, Graduate School of Business (1975).
\bibitem{M5} MScFE630 Computational Finance Module 5: Monte Carlo Methods for Risk Management

\end{thebibliography}

\end{document}